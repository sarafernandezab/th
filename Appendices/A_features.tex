\chapter{Features related with time and frequency}

\begin{enumerate}

\item Mean Absolute Value (MAV)

mav=sum(abs(f(1:end,:)))/win; %abs()

\item Mean Absolute Value Slope (MAVS)

mavs=[mav(:,2:end)- mav(:,1:end-1) zeros(size(mav,1),1)];

\item Zero Crossing (ZC)

zc=sum((f(2:end,:).*f(1:end-1,:))<=0)/(win-1);

\item Power RMS (Prms) 

$prms=sqrt(sum(f.^2))/win;$

\item Waveform of Curve Length (WL) 

wl=sum(abs(f(2:end,:)-f(1:end-1,:)));

\item Variance (var)

varf=var(f,[],1);

\item Amplitude Distribution Kurtosis (AKUR)?????

kurt=kurtosis(f); %%PROBLEMAS AQUI COM NaN

\item \textbf{Amplitude distribution Skewness (ASKW)????}

skw=skewness(f);  %%PROBLEMAS AQUI COM NaN

\item \textbf{Crest factor (CRF)}

relation between peak amplitude and rms of the signal.
crf=(max(f(:,1:end))-min(f(:,1:end)))/2.*prms(1,1:end);

\item \textbf{Teager energy (TE)}
(Rajpurohit, 2014)
for i=1:size(f,2)
     Teager energy (Rajpurohit, 2014)
    $teaE(1,i)=(sum(f(2:end-1,i).^2-f(1:end-2,i).*f(3:end,i)))/(length(f)-2);$
    % Peaks (Rajpurohit, 2014)
    %peaks(1,i)=numel(findpeaks(f(:,i),par.fsOut));

% Noise (Rajpurohit, 2014)
%three times the std of the amplitudes in the data window
\item \textbf{Standard deviation of noise (?NAME?)}%9.
estimação do std do ruído (é o scale parameter da pdf)
calcular hilbert transf.
$fi=hilbert(f);
f_env=abs(f+fi*sqrt(-1));$
sdt do ruído estimado com mediana: mediana=sP*sqrt(2*ln2)
$spn=median(f_env)/sqrt(2*log(2));$


\end{enumerate}

\section{Frequency related features}
